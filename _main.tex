% Options for packages loaded elsewhere
\PassOptionsToPackage{unicode}{hyperref}
\PassOptionsToPackage{hyphens}{url}
%
\documentclass[
  openany]{book}
\usepackage{lmodern}
\usepackage{amssymb,amsmath}
\usepackage{ifxetex,ifluatex}
\ifnum 0\ifxetex 1\fi\ifluatex 1\fi=0 % if pdftex
  \usepackage[T1]{fontenc}
  \usepackage[utf8]{inputenc}
  \usepackage{textcomp} % provide euro and other symbols
\else % if luatex or xetex
  \usepackage{unicode-math}
  \defaultfontfeatures{Scale=MatchLowercase}
  \defaultfontfeatures[\rmfamily]{Ligatures=TeX,Scale=1}
\fi
% Use upquote if available, for straight quotes in verbatim environments
\IfFileExists{upquote.sty}{\usepackage{upquote}}{}
\IfFileExists{microtype.sty}{% use microtype if available
  \usepackage[]{microtype}
  \UseMicrotypeSet[protrusion]{basicmath} % disable protrusion for tt fonts
}{}
\makeatletter
\@ifundefined{KOMAClassName}{% if non-KOMA class
  \IfFileExists{parskip.sty}{%
    \usepackage{parskip}
  }{% else
    \setlength{\parindent}{0pt}
    \setlength{\parskip}{6pt plus 2pt minus 1pt}}
}{% if KOMA class
  \KOMAoptions{parskip=half}}
\makeatother
\usepackage{xcolor}
\IfFileExists{xurl.sty}{\usepackage{xurl}}{} % add URL line breaks if available
\IfFileExists{bookmark.sty}{\usepackage{bookmark}}{\usepackage{hyperref}}
\hypersetup{
  pdftitle={A Mini Bookdown Test OER},
  pdfauthor={Tim Ream},
  hidelinks,
  pdfcreator={LaTeX via pandoc}}
\urlstyle{same} % disable monospaced font for URLs
\usepackage{longtable,booktabs}
% Correct order of tables after \paragraph or \subparagraph
\usepackage{etoolbox}
\makeatletter
\patchcmd\longtable{\par}{\if@noskipsec\mbox{}\fi\par}{}{}
\makeatother
% Allow footnotes in longtable head/foot
\IfFileExists{footnotehyper.sty}{\usepackage{footnotehyper}}{\usepackage{footnote}}
\makesavenoteenv{longtable}
\usepackage{graphicx,grffile}
\makeatletter
\def\maxwidth{\ifdim\Gin@nat@width>\linewidth\linewidth\else\Gin@nat@width\fi}
\def\maxheight{\ifdim\Gin@nat@height>\textheight\textheight\else\Gin@nat@height\fi}
\makeatother
% Scale images if necessary, so that they will not overflow the page
% margins by default, and it is still possible to overwrite the defaults
% using explicit options in \includegraphics[width, height, ...]{}
\setkeys{Gin}{width=\maxwidth,height=\maxheight,keepaspectratio}
% Set default figure placement to htbp
\makeatletter
\def\fps@figure{htbp}
\makeatother
\setlength{\emergencystretch}{3em} % prevent overfull lines
\providecommand{\tightlist}{%
  \setlength{\itemsep}{0pt}\setlength{\parskip}{0pt}}
\setcounter{secnumdepth}{5}

\title{A Mini Bookdown Test OER}
\author{Tim Ream}
\date{}

\begin{document}
\maketitle

{
\setcounter{tocdepth}{1}
\tableofcontents
}
\hypertarget{create-your-open-textbook}{%
\chapter{Create your open textbook}\label{create-your-open-textbook}}

Hello!!

\hypertarget{introduction}{%
\chapter{Introduction}\label{introduction}}

Hello. My name is \href{https://timream.net}{Tim Ream} and I'm the \emph{Systems Librarian} at \textbf{Fullerton College}. I hope you enjoy my sample OER book.

My sample OER book will include chapters related to being a functioning \emph{Systems Librarian} at the community college level.

\hypertarget{library-website}{%
\chapter{Library Website}\label{library-website}}

Almost every community college Systems Librarian will have to edit, manage, and develop their library's website. A good library website will provide information and resources to your primary users
+ Students
+ Faculty
+ Community
Poor library websites are sometimes \textbf{too focused on meeting the needs of librarians and staff, and contain terminology and processes foreign to your primary audiences.}

\hypertarget{integrated-library-system}{%
\chapter{Integrated Library System}\label{integrated-library-system}}

Aside from your regular librarian duties (which might involve instruction, reference, collection development, etc.), your primary focus as a community college Systems Librarian will involve managing the Integrated Library System (ILS). The ILS is what allows your library to process materials and make them publicly available for searching. As various systems have become more sophisticated and comprehensive, they are now sometimes referred to as a Library Services Platform (LSP). The LSP will allow your library added functionality with Acquistions, Discovery, and Analytics.

Typically your ILS or LSP will be provided by a commercial vendor such as:
+ Ex Libris
+ Sirsi-Dynix
+ Sierra

\end{document}
